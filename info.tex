\documentclass[a4paper,10pt]{article}
%\documentclass[a4paper,10pt]{scrartcl}

\usepackage[utf8]{inputenc}
\usepackage{amsmath,amsfonts,graphicx,enumerate,tikz,url,amssymb,epsfig,color,xspace,latexsym}
\usepackage[pdftitle={Explanation to vigenere},pdfauthor={Samir Musali}]{hyperref}

\begin{document}
\title{\textbf{Explanation to vigenere}}
\author{Samir Musali}
\date{January 19, 2015}
\maketitle
\providecommand{\abs}[1]{\lvert#1\rvert}

\pdfinfo{%
  /Title    (Explanation to vigenere)
  /Author   (Samir Musali)
  /Creator  (Samir Musali)
}

Here are the 3 steps of my solution:
\begin{enumerate}
 \item \textbf{Guessing the key-length:}
 Here I used shifting method. As a shifting, I used right-shifting. After each shifting process, I am
counting the letters that are in the same position as they were before shifting. After several shifting, I
am finding the shifting numbers in which the repetition was the highest. After found, we can easily see
the pattern, and guess the keylength. I tried to find the exact \textit{key-length}, and in the cases that I tested, it
was successful: I found the highest repetition number, say $n$, then I listed the shifting numbers in which the 
repetition number was more than or equal to $\frac{9n}{10}$, then I found the gcd of those listed numbers, and 
it was key-length. I coded it to automatically find the \textit{key-length} as well but also gave the chance 
to reenter the shift number if user wants to make the another guess;
  \item \textbf{Finding the key:}
  It is the most interesting part of the story. I used here numerical method. First, from the following
link I got the frequency of English letters: \href{http://en.wikipedia.org/wiki/Letter_frequency}{Letter frequency}.
\\Then I created the global array consisting of frequency of letters according to that given frequency. I
multiplied the appropriate frequency percentage by $1000$, and rounded for being integer; e.g. \textit{'b'} has
frequency $1.492$ percent, in my array the appropriate value is $1492$. And the total sum of the elements of the
global array is $99,999$, as you see in my source codes.
\\After finding the \textit{key-length}, I divided the ciphertext into the block having the size of \textit{key-length}. 
Each block has the columns numbered from $0,1,...,(key-length-1)$; so, I grouped the letters in the $k$th column of 
each block, then in each group I found the letter frequency for all members of the group. Let's name the groups as 
$W_0, W_1,..., W_{key-length-1}$. Now I will compare the frequency groups with the standart English letter frequency
table. For this I used residual methods, but for this I need to adjust the group members' frequencies array that I created;
i.e. it should have the same total sum with others for simplicity. For this purpose, I did $\frac{99999W_j\left(i\right)}{\left|W_j\right|}
\forall i\in\left[0,25\right], \forall j\in\left[0,\left(key-length-1\right)\right]$ with rounding for integer.
\\Then I found the residuals for each group; e.g. for $W_0$ I did $residual_0=
\\=\sum_{i=0}^{25} \left(A(i)-W_0(i)\right)^2$; i.e. summed up all squares, where $A$ is the standard English letter frequency array.
\\Then I shifted the groups I am tested for refinding residual; so, for each group I
put all $26$ residual values in vector after shifting. Then I find the minimum shift number in which the
residual is minimum; that number is the corresponding \textit{key-letter};
  \item \textbf{Decrypting:} 
  Follow the previous parts, and find the key. Then divided the text into the blocks each of which has length of \textit{key-length}. Then subtract the key from each block.
After all combine all subtracted parts in order to get the decrypted message.
\end{enumerate}
\centering\textbf{Thank you for taking a look at my explanation!}
\end{document}
